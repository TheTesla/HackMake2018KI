% Copyright 2004 by Till Tantau <tantau@users.sourceforge.net>.
%
% In principle, this file can be redistributed and/or modified under
% the terms of the GNU Public License, version 2.
%
% However, this file is supposed to be a template to be modified
% for your own needs. For this reason, if you use this file as a
% template and not specifically distribute it as part of a another
% package/program, I grant the extra permission to freely copy and
% modify this file as you see fit and even to delete this copyright
% notice. 

\documentclass{beamer}
\usepackage[ngerman]{babel}
\usepackage[utf8]{inputenc}
\usepackage{graphics}
\usepackage{tikz}
\usetikzlibrary{positioning}
\usepackage[decimalsymbol=comma]{siunitx}
\usepackage{amsmath}
\usepackage{ziffer}
\usepackage{scrextend}
\usepackage{color, colortbl}
\usepackage{pgf-pie}


% There are many different themes available for Beamer. A comprehensive
% list with examples is given here:
% http://deic.uab.es/~iblanes/beamer_gallery/index_by_theme.html
% You can uncomment the themes below if you would like to use a different
% one:
%\usetheme{AnnArbor}
%\usetheme{Antibes}
%\usetheme{Bergen}
%\usetheme{Berkeley}
%\usetheme{Berlin}
%\usetheme{Boadilla}
%\usetheme{boxes}
%\usetheme{CambridgeUS}
%\usetheme{Copenhagen}
%\usetheme{Darmstadt}
%\usetheme{default}
%\usetheme{Frankfurt}
%\usetheme{Goettingen}
%\usetheme{Hannover}
%\usetheme{Ilmenau}
%\usetheme{JuanLesPins}
%\usetheme{Luebeck}
%\usetheme{Madrid}
%\usetheme{Malmoe}
%\usetheme{Marburg}
%\usetheme{Montpellier}
%\usetheme{PaloAlto}
\usetheme{Pittsburgh}
%\usetheme{Rochester}
%\usetheme{Singapore}
%\usetheme{Szeged}
%\usetheme{Warsaw}

\newcommand*\conj[1]{\overline{#1}}
\newcommand*\oo{\infty}

\definecolor{tucinfo}{rgb}{0.49,0.6,0.16}
\definecolor{LightCyan}{rgb}{0.88,1,1}

\setbeamercolor{title}{fg=blue}
\setbeamercolor{structure}{fg=blue}


\title{Künstliche Intelligenz -- Wie geht das denn?}

% A subtitle is optional and this may be deleted
\subtitle{Methoden und Komponenten am Beispiel der Bildverarbeitung}

\author{Stefan Helmert}
% - Give the names in the same order as the appear in the paper.
% - Use the \inst{?} command only if the authors have different
%   affiliation.

\institute[TU Chemnitz] % (optional, but mostly needed)
{
  TU Chemnitz -- Professorship of Computer Graphics and Visualization
  
}

\institute[Chaostreff Chemnitz] % (optional, but mostly needed)
{
	Chaostreff Chemnitz e.V.
	
}

% - Use the \inst command only if there are several affiliations.
% - Keep it simple, no one is interested in your street address.

%\date{24.06.2017}
\date{3./4. November 2018}
%\date{13th September 2018}
% - Either use conference name or its abbreviation.
% - Not really informative to the audience, more for people (including
%   yourself) who are reading the slides online

\subject{Bildverarbeitung}
% This is only inserted into the PDF information catalog. Can be left
% out. 

% If you have a file called "university-logo-filename.xxx", where xxx
% is a graphic format that can be processed by latex or pdflatex,
% resp., then you can add a logo as follows:

% \pgfdeclareimage[height=0.5cm]{university-logo}{university-logo-filename}
% \logo{\pgfuseimage{university-logo}}

% Delete this, if you do not want the table of contents to pop up at
% the beginning of each subsection:
%\AtBeginSubsection[]
%{
%  \begin{frame}<beamer>{Inhalt}
%    \tableofcontents[currentsection,currentsubsection]
%  \end{frame}
%}

% Let's get started
\begin{document}

\begin{frame}
  \titlepage
\end{frame}

\begin{frame}{Inhalt}
  \tableofcontents
  % You might wish to add the option [pausesections]
\end{frame}

% Section and subsections will appear in the presentation overview
% and table of contents.
\pagenumbering{arabic}

\section{Was wollen wir?}
\subsection{Anwendungen}
\begin{frame}{\insertsection}{\insertsubsection}
\begin{itemize}
	\item Bildersuche
	\item Suche in Videos
	\item Klassifikation von Mediendaten
	\item Analyse der Wirkung und Stimmung von Medien
	\item Verbesserung
	\begin{itemize}
		\item Monochrom $\rightarrow$ Farbe
		\item SD $\rightarrow$ HD
	\end{itemize}
	\item Erstellung neuer Inhalte
\end{itemize}
\end{frame}

\subsection{Autonomie}
\begin{frame}{\insertsection}{\insertsubsection}
	\begin{itemize}
		\item Lernen statt Programmieren
		\item Keine (einschränkenden) Vorgaben
		\item Funktion ohne Vorwissen
		\item Eigenständige Verbesserung
		\item Effizienz
	\end{itemize}
\end{frame}

\section{Umsetzung}
\subsection{Da gibt es doch was auf Github}
\begin{frame}{\insertsection}{\insertsubsection}
	\begin{tikzpicture}[
	roundnode/.style={circle, draw=green!60, fill=green!5, very thick, minimum size=7mm},
	squarednode/.style={rectangle, draw=red!60, fill=red!5, very thick, minimum size=5mm},
	]
	%Nodes
	\node[roundnode]   (in)                        {\shortstack{Eingabedaten}};
	\node[squarednode] (ki)    [right=of in]       {KI};
	\node[roundnode]   (label) [right=of ki] {\shortstack{Ergebnis}};
	%Lines
	\draw[->] (in.east) --  (ki.west);
	\draw[->] (ki.east) --  (label.west);

	\end{tikzpicture}
\end{frame}

\subsection{Überwachtes Lernen}
\begin{frame}{\insertsection}{\insertsubsection}
\begin{tikzpicture}[
roundnode/.style={circle, draw=green!60, fill=green!5, very thick, minimum size=7mm},
squarednode/.style={rectangle, draw=red!60, fill=red!5, very thick, minimum size=5mm},
]
%Nodes
\node[roundnode]   (in)                        {\shortstack{bekannte \\ Rohdaten}};
\node[roundnode]   (label) [above=of in] {\shortstack{erwartete \\ Ausgabe}};
\node[squarednode] (ki)    [right=of in]       {KI};
\node[squarednode] (eq)    [above right=of ki] {=};

%Lines
\draw[->] (label.east) .. controls +(right:30mm) and +(up:10mm) .. (eq.north);
\draw[->] (in.east) --                                             (ki.west);
\draw[->] (ki.east) .. controls +(right:10mm) and +(down:10mm) ..  (eq.south);
\draw[->] (eq.west) .. controls +(left:10mm) and +(up:10mm) ..     (ki.north) node[midway,sloped] {\shortstack{update\bigskip}} ;
\end{tikzpicture}
\end{frame}

\subsection{Unüberwachtes Lernen}
\begin{frame}{\insertsection}{\insertsubsection}
	\begin{tikzpicture}[
	roundnode/.style={circle, draw=green!60, fill=green!5, very thick, minimum size=7mm},
	squarednode/.style={rectangle, draw=red!60, fill=red!5, very thick, minimum size=5mm},
	]
	%Nodes
	\node[roundnode]   (in)                        {\shortstack{bekannte \\ Rohdaten}};
	\node[squarednode] (dec)    [right=of in]       {KI-Decoder};
	\node[squarednode] (eq)    [above=of dec] {=};
	\node[squarednode] (enc) [right=of dec] {\shortstack{KI-Encoder}};	
	%Lines
	\draw[->] (dec.east) -- (enc.west);
	\draw[->] (in.east) --                                             (dec.west);
	\draw[->] (in.east) .. controls +(right:10mm) and +(left:10mm) .. (eq.west);
	\draw[->] (enc.east) .. controls +(right:10mm) and +(right:50mm) ..  (eq.east);
	\draw[->] (eq.south) .. controls +(down:10mm) and +(up:10mm) ..     (enc.north) node[midway,sloped] {\shortstack{update\bigskip}} ;
	\draw[->] (enc.south) .. controls +(down:15mm) and +(down:15mm) ..     (dec.south) node[midway,sloped] {\shortstack{update\bigskip}} ;
	\end{tikzpicture}
\end{frame}

\subsection{Backpropagation}
\begin{frame}{\insertsection}{\insertsubsection}
	\begin{tikzpicture}[
	roundnode/.style={circle, draw=green!60, fill=green!5, very thick, minimum size=7mm},
	squarednode/.style={rectangle, draw=red!60, fill=red!5, very thick, minimum size=5mm},
	]
	%Nodes
	\node[roundnode]   (in)                        {\shortstack{bekannte \\ Rohdaten}};
	\node[roundnode]   (label) [above=of in] {\shortstack{erwartete \\ Ausgabe}};
	\node[squarednode] (l1)    [right=of in]       {L1};
	\node[squarednode] (l2)    [right=of l1]       {L2};
	\node[squarednode] (l3)    [right=of l2]       {L3};
	\node[squarednode] (l4)    [right=of l3]       {L4};
	\node[squarednode] (eq)    [right=of l4] {$-$};

	%Lines
	\draw[->] (in.east) -- (l1.west);
	\draw[->] (l1.east) -- (l2.west);
	\draw[->] (l2.east) -- (l3.west);
	\draw[->] (l3.east) -- (l4.west);
	\draw[->] (l4.east) -- (eq.west);
	\draw[->] (eq.south) .. controls +(down:10mm) and +(down:10mm) ..     (l4.south) node[midway,sloped] {\shortstack{\vspace{0.5cm} \\ update}} ;
	\draw[->] (l4.north) .. controls +(up:10mm) and +(up:10mm) ..     (l3.north) node[midway,sloped] {\shortstack{update\bigskip}} ;
	\draw[->] (l3.south) .. controls +(down:10mm) and +(down:10mm) ..     (l2.south) node[midway,sloped] {\shortstack{\vspace{0.5cm} \\ update}} ;
	\draw[->] (l2.north) .. controls +(up:10mm) and +(up:10mm) ..     (l1.north) node[midway,sloped] {\shortstack{update\bigskip}} ;
	\draw[->] (label.east) .. controls +(right:70mm) and +(up:10mm) ..     (eq.north);
	\end{tikzpicture}
\end{frame}

\subsection{Neuron}
\begin{frame}{\insertsection}{\insertsubsection}
	\begin{tikzpicture}[
	nonode/.style={circle, fill=green!5, very thick, minimum size=7mm},
	roundnode/.style={circle, draw=red!60, fill=red!5, very thick, minimum size=11mm},
	squarednode/.style={rectangle, draw=red!60, fill=red!5, very thick, minimum size=5mm},
	]
	%Nodes
	\node[nonode]    (ina)                                         {a};		
	\node[nonode]    (inb)    [below=of ina]                       {b};
	\node[nonode]    (inc)    [below=of inb]                       {c};
	\node[roundnode] (neuron) [right=of inb]                       {+};
	\node[nonode]    (out)    [right=of neuron]                    {out};
	%Lines
	\draw[->] (ina.east) .. controls +(right:10mm) and +(up:10mm) .. (neuron.north) node[midway,sloped] {\shortstack{$-1$\bigskip}};
	\draw[->] (inb.east) -- (neuron.west)  node[midway,sloped] {\shortstack{$+2$\bigskip}};
	\draw[->] (inc.east) .. controls +(right:10mm) and +(down:10mm) .. (neuron.south) node[midway,sloped] {\shortstack{$-1$\bigskip}};	
	\draw[->] (neuron.east) -- (out.west);
	\end{tikzpicture}
	
	Was fehlt noch?
\end{frame}

\begin{frame}{\insertsection}{\insertsubsection}
	\begin{tikzpicture}[
	nonode/.style={circle, fill=green!5, very thick, minimum size=7mm},
	roundnode/.style={circle, draw=red!60, fill=red!5, very thick, minimum size=11mm},
	squarednode/.style={rectangle, draw=red!60, fill=red!5, very thick, minimum size=5mm},
	]
	%Nodes
	\node[nonode]    (ina)                                         {a};		
	\node[nonode]    (inb)    [below=of ina]                       {b};
	\node[nonode]    (inc)    [below=of inb]                       {c};	\node[nonode]    (c)      [below=of inc]                       {+1};
	\node[roundnode] (neuron) [right=of inb]                       {+};
	\node[nonode]    (out)    [right=of neuron]                    {out};
	%Lines
	\draw[->] (ina.east) .. controls +(right:10mm) and +(up:10mm) .. (neuron.north) node[midway,sloped] {\shortstack{$-1$\bigskip}};
	\draw[->] (inb.east) -- (neuron.west)  node[midway,sloped] {\shortstack{$+2$\bigskip}};
	\draw[->] (inc.east) .. controls +(right:10mm) and +(down:10mm) .. (neuron.south) node[midway,sloped] {\shortstack{$-1$\bigskip}};
	\draw[->] (c.east) .. controls +(right:10mm) and +(down:10mm) .. (neuron.south) node[midway,sloped] {\shortstack{$0$\bigskip}};	
	\draw[->] (neuron.east) -- (out.west);
	\end{tikzpicture}
\end{frame}

\subsection{Faltungsnetz}
\begin{frame}{\insertsection}{\insertsubsection}
	\begin{tikzpicture}[
	invisiblenode/.style={minimum size=5mm},
	slidenode/.style={rectangle, draw=blue!60, fill=blue!5, very thick, minimum size=6mm},
	nonode/.style={circle, fill=green!5, very thick, minimum size=7mm},
	roundnode/.style={circle, draw=red!60, fill=red!5, very thick, minimum size=11mm},
	squarednode/.style={rectangle, draw=red!60, fill=red!5, very thick, minimum size=5mm},
	]
	%Nodes
	\node[slidenode]    (slide1)                {5};
	\node[slidenode]    (slide2) [right=of slide1]                  {5};	
	\node[slidenode]    (slide3) [right=of slide2]                  {1};
	\node[slidenode]    (slide4) [right=of slide3]               {6};	
	\node[slidenode]    (slide5) [right=of slide4]               {8};	
	\node[slidenode]    (slide6) [right=of slide5]               {9};
	\node[roundnode] (neuron) [below=of slide2]                       {+};	\node[invisiblenode]    (arl) [right=of neuron]               {};
	\node[nonode]    (out)    [below=of neuron]                    {out};		\node[invisiblenode]    (arr) [right=of arl]               {};
	%Lines
	\draw[->] (slide3.south) .. controls +(down:10mm) and +(right:10mm) .. (neuron.east) node[midway,sloped] {\shortstack{$-1$\bigskip}};
	\draw[->] (slide2.south) -- (neuron.north)  node[midway,sloped] {\shortstack{$+2$\bigskip}};
	\draw[->] (slide1.south) .. controls +(down:10mm) and +(left:10mm) .. (neuron.west) node[midway,sloped] {\shortstack{$-1$\bigskip}};	
	\draw[->] (neuron.south) -- (out.north);
	\draw[->,thick] (arl.east) -- (arr.west);
	\end{tikzpicture}
	
	Und bei Farbbildern?
\end{frame}

\begin{frame}{\insertsection}{\insertsubsection}
	\begin{tikzpicture}[
	invisiblenode/.style={minimum size=5mm},
	slidenodeb/.style={rectangle, draw=blue!60, fill=blue!5, very thick, minimum size=6mm},
	slidenodeg/.style={rectangle, draw=green!60, fill=green!5, very thick, minimum size=6mm},
	slidenoder/.style={rectangle, draw=red!60, fill=red!5, very thick, minimum size=6mm},
	nonode/.style={circle, fill=green!5, very thick, minimum size=7mm},
	roundnode/.style={circle, draw=red!60, fill=red!5, very thick, minimum size=11mm},
	squarednode/.style={rectangle, draw=red!60, fill=red!5, very thick, minimum size=5mm},
	]
	%Nodes
	\node[slidenodeb]    (slide1b)                {5};
	\node[slidenodeb]    (slide2b) [below=of slide1b]                  {5};	
	\node[slidenodeb]    (slide3b) [below=of slide2b]                  {1};
	\node[slidenodeb]    (slide4b) [below=of slide3b]               {6};	
	\node[slidenodeg]    (slide1g) [left=of slide1b]               {5};
	\node[slidenodeg]    (slide2g) [left=of slide2b]                  {5};	
	\node[slidenodeg]    (slide3g) [left=of slide3b]                  {1};
	\node[slidenodeg]    (slide4g) [left=of slide4b]               {6};	
	\node[slidenoder]    (slide1r) [left=of slide1g]               {5};
	\node[slidenoder]    (slide2r) [left=of slide2g]                  {5};	
	\node[slidenoder]    (slide3r) [left=of slide3g]                  {1};
	\node[slidenoder]    (slide4r) [left=of slide4g]               {6};	

	\node[roundnode] (neuron) [right=of slide2b]                       {+};	\node[invisiblenode]    (arl) [below=of neuron]               {};
	\node[nonode]    (out)    [right=of neuron]                    {out};		\node[invisiblenode]    (arr) [below=of arl]               {};
	%Lines
	\draw[->] (slide3b.east) .. controls +(right:10mm) and +(down:10mm) .. (neuron.south) ;
	\draw[->] (slide3r.north) .. controls +(up:5mm) and +(down:10mm) .. (neuron.south) ;	
	\draw[->] (slide3g.south) .. controls +(down:10mm) and +(down:20mm) .. (neuron.south) ;	

	\draw[->] (slide2b.east) -- (neuron.west);
	\draw[->] (slide2r.north) .. controls +(up:5mm) and +(up:3mm) .. (neuron.north) ;	
	\draw[->] (slide2g.south) .. controls +(down:10mm) and +(down:3mm) .. (neuron.south) ;	
	\draw[->] (slide1b.east) .. controls +(right:10mm) and +(up:10mm) .. (neuron.north) ;
	\draw[->] (slide1r.south) .. controls +(down:5mm) and +(up:10mm) .. (neuron.north) ;	
	\draw[->] (slide1g.north) .. controls +(up:10mm) and +(up:20mm) .. (neuron.north) ;	
	\draw[->] (neuron.east) -- (out.west);
	\draw[->,thick] (arl.south) -- (arr.north);
	\end{tikzpicture}
	

\end{frame}

\begin{frame}{\insertsection}{\insertsubsection}
	\begin{tikzpicture}[
	invisiblenode/.style={minimum size=5mm},
	slidenode/.style={rectangle, draw=blue!60, fill=blue!5, very thick, minimum size=6mm},
	nonode/.style={circle, fill=green!5, very thick, minimum size=7mm},
	roundnode/.style={circle, draw=red!60, fill=red!5, very thick, minimum size=11mm},
	squarednode/.style={rectangle, draw=red!60, fill=red!5, very thick, minimum size=5mm},
	]
	%Nodes
	\node[slidenode]    (slide1)                {5};
	\node[slidenode]    (slide2) [right=of slide1]                  {5};	
	\node[slidenode]    (slide3) [right=of slide2]                  {1};
	\node[slidenode]    (slide4) [right=of slide3]               {6};	
	\node[slidenode]    (slide5) [right=of slide4]               {8};	
	\node[slidenode]    (slide6) [right=of slide5]               {9};
	\node[roundnode] (neuron) [below=of slide2]                       {+};	\node[invisiblenode]    (arl) [right=of neuron]               {};
	\node[invisiblenode]    (arr) [right=of arl]               {};

	\node[slidenode]    (slide2o) [below=of neuron]                  {-1};
	\node[slidenode]    (slide1o) [left=of slide2o]               {1};	
	\node[slidenode]    (slide3o) [right=of slide2o]                  {1};
	\node[slidenode]    (slide4o) [right=of slide3o]               {0};	
	\node[slidenode]    (slide5o) [right=of slide4o]               {-3};	
	\node[slidenode]    (slide6o) [right=of slide5o]               {2};
	%Lines
	\draw[->] (slide3.south) .. controls +(down:10mm) and +(right:10mm) .. (neuron.east) node[midway,sloped] {\shortstack{?\bigskip}};
	\draw[->] (slide2.south) -- (neuron.north)  node[midway,sloped] {\shortstack{?\bigskip}};
	\draw[->] (slide1.south) .. controls +(down:10mm) and +(left:10mm) .. (neuron.west) node[midway,sloped] {\shortstack{?\bigskip}};	
	\draw[-] (neuron.south) -- (slide2o.north) node[midway,sloped] {\shortstack{$=$\bigskip}};
	\draw[->,thick] (arl.east) -- (arr.west);
	\end{tikzpicture}
\end{frame}


\subsection{Deeplearning}
\begin{frame}{\insertsection}{\insertsubsection}
	\begin{tikzpicture}[
	nonode/.style={circle, fill=green!5, very thick, minimum size=7mm},
	roundnode/.style={circle, draw=red!60, fill=red!5, very thick, minimum size=11mm},
	squarednode/.style={rectangle, draw=red!60, fill=red!5, very thick, minimum size=5mm},
	]
	%Nodes
	\node[roundnode]    (ina)                                         {+};		
	\node[roundnode]    (inb)    [below=of ina]                       {+};
	\node[roundnode]    (inc)    [below=of inb]                       {+};
	\node[roundnode] (neuron) [right=of inb]                       {+};
	\node[nonode]    (out)    [right=of neuron]                    {out};
	%Lines
	\draw[->] (ina.east) .. controls +(right:10mm) and +(up:10mm) .. (neuron.north) node[midway,sloped] {\shortstack{$-1$\bigskip}};
	\draw[->] (inb.east) -- (neuron.west)  node[midway,sloped] {\shortstack{$+2$\bigskip}};
	\draw[->] (inc.east) .. controls +(right:10mm) and +(down:10mm) .. (neuron.south) node[midway,sloped] {\shortstack{$-1$\bigskip}};	
	\draw[->] (neuron.east) -- (out.west);
	\end{tikzpicture}
\end{frame}

\begin{frame}{\insertsection}{\insertsubsection}
	\begin{tikzpicture}[
	invisiblenode/.style={minimum size=5mm},
	slidenode/.style={rectangle, draw=blue!60, fill=blue!5, very thick, minimum size=6mm},
	nonode/.style={circle, fill=green!5, very thick, minimum size=7mm},
	roundnode/.style={circle, draw=red!60, fill=red!5, very thick, minimum size=11mm},
	squarednode/.style={rectangle, draw=red!60, fill=red!5, very thick, minimum size=5mm},
	]
	%Nodes
	\node[roundnode] (neuron0)                       {+};	
	\node[slidenode]    (slide2) [below=of neuron0]  {?};	
	\node[slidenode]    (slide1) [left=of slide2]    {?};
	\node[slidenode]    (slide3) [right=of slide2]   {?};
	\node[slidenode]    (slide4) [right=of slide3]   {?};	
	\node[slidenode]    (slide5) [right=of slide4]   {?};	
	\node[slidenode]    (slide6) [right=of slide5]   {?};
	\node[roundnode] (neuron) [below=of slide2]      {+};
	\node[invisiblenode]    (arl) [right=of neuron]  {};
	\node[invisiblenode]    (arr) [right=of arl]     {};
	\node[slidenode]    (slide2o) [below=of neuron]  {?};
	\node[slidenode]    (slide1o) [left=of slide2o]  {?};	
	\node[slidenode]    (slide3o) [right=of slide2o] {?};
	\node[slidenode]    (slide4o) [right=of slide3o] {?};	
	\node[slidenode]    (slide5o) [right=of slide4o] {?};	
	\node[slidenode]    (slide6o) [right=of slide5o] {?};
	\node[roundnode] (neuron2) [below=of slide2o]    {+};	
	%Lines
		\draw[-] (neuron0.south) -- (slide2.north) node[midway,sloped] {\shortstack{$=$\bigskip}};
	\draw[->] (slide3.south) .. controls +(down:10mm) and +(right:10mm) .. (neuron.east) node[midway,sloped] {\shortstack{?\bigskip}};
	\draw[->] (slide2.south) -- (neuron.north)  node[midway,sloped] {\shortstack{?\bigskip}};
	\draw[->] (slide1.south) .. controls +(down:10mm) and +(left:10mm) .. (neuron.west) node[midway,sloped] {\shortstack{?\bigskip}};	
	\draw[-] (neuron.south) -- (slide2o.north) node[midway,sloped] {\shortstack{$=$\bigskip}};
	\draw[->,thick] (arl.east) -- (arr.west);
		\draw[->] (slide3o.south) .. controls +(down:10mm) and +(right:10mm) .. (neuron2.east) node[midway,sloped] {\shortstack{?\bigskip}};
		\draw[->] (slide2o.south) -- (neuron2.north)  node[midway,sloped] {\shortstack{?\bigskip}};
		\draw[->] (slide1o.south) .. controls +(down:10mm) and +(left:10mm) .. (neuron2.west) node[midway,sloped] {\shortstack{?\bigskip}};	
		
	\end{tikzpicture}
\end{frame}

\subsection{XOR-Problem}
\begin{frame}{\insertsection}{\insertsubsection}
$\begin{array}{cccccc}
a	& b  & \mathrm{not}(a) & \mathrm{or}(a,b) & \mathrm{and}(a,b)  & \mathrm{xor}(a,b) \\ 
0	& 0  & 1      & 0 & 0 & 0 \\ 
0	& 1  & 1      & 1 & 0 & 1 \\ 
1	& 0  & 0      & 1 & 0 & 1 \\ 
1	& 1  & 0      & 1 & 1 & 0
\end{array} $
\end{frame}

\begin{frame}{\insertsection}{\insertsubsection}
	$\begin{array}{cccccc}
	a	& b  & \mathrm{not}(a) & \mathrm{or}(a,b) & \mathrm{and}(a,b)  & \mathrm{xor}(a,b) \\ 
	0	& 0  & 1      & 0 & 0 & 0 \\ 
	0	& 1  & 1      & 1 & 0 & 1 \\ 
	1	& 0  & 0      & 1 & 0 & 1 \\ 
	1	& 1  & 0      & 1 & 1 & 0
	\end{array} $
	$$\mathrm{xor}(a,b) = \mathrm{and}(\mathrm{or}(a,b), \mathrm{not}(\mathrm{and}(a,b))) $$
	
	$$\mathrm{xor}(a,b) = \mathrm{or}(a,b) - \mathrm{and}(a,b) $$
\end{frame}

\begin{frame}{\insertsection}{\insertsubsection}
	$$\mathrm{xor}(a,b) = \mathrm{or}(a,b) - \mathrm{and}(a,b) $$

$$\mathrm{not}(a) = 1-a ; \mathrm{or}(a,b) = a+b ; \mathrm{and}(a,b) = \frac{a+b}{2}$$
\bigskip

	$\begin{array}{cccccc}
	a	& b  & \mathrm{not}(a) & \mathrm{or}(a,b) & \mathrm{and}(a,b)  & \mathrm{xor}(a,b) \\ 
	0	& 0  & 1      & 0 & 0 & 0 \\ 
	0	& 1  & 1      & 1 & 0,5 & 0,5 \\ 
	1	& 0  & 0      & 1 & 0,5 & 0,5 \\ 
	1	& 1  & 0      & 2 & 1 & 1
	\end{array} $


\end{frame}

\begin{frame}{\insertsection}{\insertsubsection}
	
	$$\mathrm{and}(a,b) = a\cdot b$$

	$\rightarrow$ Nein, das ist nicht linear.
	
\end{frame}

\begin{frame}{\insertsection}{\insertsubsection}
	
	$$\mathrm{xor}(a,b) = \mathrm{and}(\mathrm{or}(a,b), \mathrm{not}(\mathrm{and}(a,b))) $$
	
	$$\mathrm{not}(a) = 1-a$$
	
	$$\mathrm{or}(a,b) = a+b$$ 
	$$\mathrm{and}(a,b) = \frac{a+b}{2}$$
	
	$$\mathrm{xor}(a,b) = \frac{\left(a + b\right) + \left(- \frac{a+b}{2}\right)}{2} = \frac{a+b}{4}$$
	
	
\end{frame}

\begin{frame}{\insertsection}{\insertsubsection}
	
	$$\mathrm{xor}(a,b) = \frac{\left(a + b\right) + \left(- \frac{a+b}{2}\right)}{2} = \frac{a+b}{4}$$	
	
	\bigskip
	
	$\begin{array}{cccccc}
	a	& b  & \mathrm{not}(a) & \mathrm{or}(a,b) & \mathrm{and}(a,b)  & \mathrm{xor}(a,b) \\ 
	0	& 0  & 1      & 0 & 0 & 0 \\ 
	0	& 1  & 1      & 1 & 0,5 & 0,25 \\ 
	1	& 0  & 0      & 1 & 0,5 & 0,25 \\ 
	1	& 1  & 0      & 2 & 1 & 0,5
	\end{array} $
	
\end{frame}


\subsection{Aktivierungsfunktion}
\begin{frame}{\insertsection}{\insertsubsection}
	
\textbf{Lösung} -- Aktivierungsfunktion
	  \begin{tikzpicture}[domain=-3.2:3.2] 
	  \draw[very thin,color=gray] (-2.1,-2.1) grid (2.9,2.9);
	  \draw[->] (-2.2,0) -- (3.2,0) node[right] {$x$}; 
	  \draw[->] (0,-2.2) -- (0,3.2) node[above] {$f(x)$};
	  \draw[color=blue]   plot (\x,{\x*(sign(\x)+1.5)/2})    node[right] {\shortstack{$f(x) = \begin{cases} 1,25\cdot x, &\text{if } x > 0 \\ 0,25 \cdot x & \end{cases}$ \bigskip }}; 
	  \draw[color=red]    plot (\x,{\x*(sign(\x)+1)/2})             node[right] {\shortstack{\bigskip \\ $f(x) = \begin{cases} x, &\text{if } x > 0 \\ 0 & \end{cases}$}}; 

	  \draw[color=green] plot (\x,{tanh(\x)}) node[right] {$f(x) = \mathrm{tanh}(x)$};
	  \draw[color=orange] plot (\x,{1/(exp(-\x)+1)}) node[right] {\shortstack{\bigskip \\ \\ $f(x) = \frac{1}{e^{-x}+1}$}};
	  \end{tikzpicture}
	  
\end{frame}

\begin{frame}{\insertsection}{\insertsubsection}
	\begin{tikzpicture}[
	nonode/.style={circle, fill=green!5, very thick, minimum size=7mm},
	roundnode/.style={circle, draw=red!60, fill=red!5, very thick, minimum size=11mm},
	squarednode/.style={rectangle, draw=red!60, fill=red!5, very thick, minimum size=5mm},
	]
	%Nodes
	\node[nonode]    (ina)                                         {a};		
	\node[nonode]    (inb)    [below=of ina]                       {b};
	\node[nonode]    (inc)    [below=of inb]                       {c};
	\node[roundnode] (neuron) [right=of inb]                       {+};
	\node[roundnode] (activation) [right=of neuron]                       {tanh};
	\node[nonode]    (out)    [right=of activation]                    {out};
	%Lines
	\draw[->] (ina.east) .. controls +(right:10mm) and +(up:10mm) .. (neuron.north) node[midway,sloped] {\shortstack{$-1$\bigskip}};
	\draw[->] (inb.east) -- (neuron.west)  node[midway,sloped] {\shortstack{$+2$\bigskip}};
	\draw[->] (inc.east) .. controls +(right:10mm) and +(down:10mm) .. (neuron.south) node[midway,sloped] {\shortstack{$-1$\bigskip}};	
	\draw[->] (activation.east) -- (out.west);
	\draw[->] (neuron.east) -- (activation.west);
	\end{tikzpicture}
	
\end{frame}

\begin{frame}{\insertsection}{\insertsubsection}
	$$\mathrm{xor}(a,b) = \mathrm{tanh}(\mathrm{or}(a,b)) - \mathrm{and}(a,b) $$
		$$\mathrm{or}(a,b) = a+b$$ 
		$$\mathrm{and}(a,b) = \frac{a+b}{2}$$

	\bigskip
	
	$\begin{array}{ccccccc}
	a	& b  & \mathrm{not}(a) & \mathrm{tanh}(\mathrm{or}(a,b)) & \mathrm{and}(a,b)  & \mathrm{xor}(a,b) & 4\cdot\mathrm{xor}(a,b) \\ 
	0	& 0  & 1      & 0 & 0 & 0 & 0 \\ 
	0	& 1  & 1      & 0,76 & 0,5 & 0,26 & 1,04 \\ 
	1	& 0  & 0      & 0,76 & 0,5 & 0,26 & 1,04\\ 
	1	& 1  & 0      & 0,96 & 1 & -0,04 & -0,16
	\end{array} $
	
	
\end{frame}

\begin{frame}{\insertsection}{\insertsubsection}
	
	Aktivierungsfunktion $\tanh(x)$ approximiert andere Funktionen.
	\begin{tikzpicture}[domain=-2.2:2.2] 
	\draw[very thin,color=gray] (-2.1,-2.1) grid (2.9,2.9);
	\draw[->] (-2.2,0) -- (3.2,0) node[right] {$x$}; 
	\draw[->] (0,-2.2) -- (0,3.2) node[above] {$f(x)$};
	\draw[domain=-2.1:1,color=blue]   plot (\x,{exp(\x)-1})    node[right] {\shortstack{$f(x) = e^{x}-1$ }}; 
	\draw[domain=-0.9:2.3,color=red]    plot (\x,{ln(\x+1)})             node[right] {\shortstack{$f(x) = \ln(x+1)$}}; 
	
	\draw[color=green] plot (\x,{tanh(\x)}) node[right] {\shortstack{\bigskip \\ $f(x) = \mathrm{tanh}(x)$}};
		\draw[color=orange] plot (\x,{\x}) node[right] {\shortstack{ $f(x) = x$}};

	\end{tikzpicture}
	
\end{frame}

\section{Praxis}
\subsection{Der Netzwerk-Graph}
\begin{frame}{\insertsection}{\insertsubsection}
	

	\begin{tikzpicture}[
	roundnode/.style={circle, draw=green!60, fill=green!5, very thick, minimum size=7mm},
	squarednode/.style={rectangle, draw=red!60, fill=red!5, very thick, minimum size=5mm},
	outernode/.style={rectangle, draw=black, very thick, minimum  width=65mm, minimum height=15mm},
	funnode/.style={rectangle, draw=black, very thick, minimum  width=15mm, minimum height=10mm},
	nonode/.style={rectangle,  minimum size=7mm},
	]
	%Nodes

	\node[roundnode]   (in)   {in};
	\node[outernode]  (o)   [right=of in]        {};
	\node[squarednode] (l1)   [right=of in]        {L1};
	\node[squarednode] (l2)    [right=of l1]       {L2};
	\node[squarednode] (l3)    [right=of l2]       {L3};
	\node[nonode] (l4)    [right=of l3]       {};
	\node[squarednode] (l4o)    [above=of l4]       {L4};
	\node[roundnode]   (out)   [right=of l4] {out};
	\node[funnode]   (f)   [below=of o] {$\mathrm{f}(x)$};
	
	%Lines
	\draw[->] (in.east) -- (l1.west);
	\draw[->] (l1.east) -- (l2.west);
	\draw[->] (l2.east) -- (l3.west);
	\draw[->] (l3.east) -- (l4.west);
	\draw[->] (l4.east) -- (out.west);
	\draw[->] (o.south) -- (f.north) node[midway, right] {compile};
	\draw[->, dashed] (l4o.south) -- (l4.north);
	\end{tikzpicture}
\end{frame}

\subsection{Aufwand}
\begin{frame}{\insertsection}{\insertsubsection}
\begin{tikzpicture}[scale=0.9]
\pie[
/tikz/every pin/.style={align=center},
text=pin,
rotate=0,
explode=0.0,
color={red!70,green!70},
]
{
	80/Datenerverarbeitung,
	20/KI
}
\end{tikzpicture}
\end{frame}


\section{Fragen?}
\begin{frame}{\insertsection}{\insertsubsection}
	???
\end{frame}

\end{document}




